% プリアンブル

%文字の大きさ、用紙の大きさ、段組
\documentclass[dvipdfmx]{jsarticle}
\bibliographystyle{jplain}

\usepackage{url}
\usepackage[dvipdfmx]{graphicx}
\usepackage{here}
\usepackage{listings, jlisting}

%表、図の番号形式
\def\thetable{\thesubsubsection.\arabic{table}}
\def\thefigure{\thesubsubsection.\arabic{figure}}

%間隔
\setlength\floatsep{0pt}
\setlength\textfloatsep{0pt}
\setlength\intextsep{0pt}
\setlength\abovecaptionskip{0pt}

\usepackage{geometry}
\usepackage{amsmath}
%余白
\geometry{left=10mm,right=10mm,top=10mm,bottom=15mm}

%画像フォルダの指定
\graphicspath{{../src/}}

\title{\vspace{-10mm}タイトル}
\author {名前1 \and 名前2}
\date {1999/12/31}

\begin{document}
    %タイトルの出力
    \maketitle

    \vspace{-15mm}
    \begin{abstract}
        うんち!!w
        \vspace{3mm}
    \end{abstract}

    \vspace{1mm}

    \section{セクション名}\label{seq:s1}
    \subsection{サブセクション名}
    \subsubsection{サブサブセクション名}
    %表の例
    \begin{table}[H]%[H]でその場に矯正出力する
        \begin{flushleft}
            \caption{表のキャプションは上\label{tb:t1}}
            \begin{tabular}{|r|r|r|r|r|}  \hline
            試行 & 段階 & 測定値[kg] & \%MAX & Grading誤差[\%] \\ \hline \hline
            1.14 & 5.14 & 19.19 & 81.0 & 9.31 \\ \hline
            893 & 45.45 & 3777 &  & 364364 \\ \hline
            \end{tabular}
            \begin{tabular}{|r|r|r|r|} \hline \hline
            & 段階 & 測定値[kg] & (測定値) - (閾値) \\ \hline \hline
            & 2001 & 7 & 20 \\ \cline{2-4}
            & 334 & 301 & 707 \\ \cline{2-4}
            平均 & 1008 & 1111 & 1783 \\ \cline{2-4}
            & 40298 & 3000000 & 42582059 \\ \cline{2-4}
            & 080 & 300.6 & 377.6 \\ \hline
            \end{tabular}
            \begin{tabular}{|r|r|} \hline \hline
            閾値[kg] & 5.5 \\ \hline \hline
            1強度あたりのグレーディング誤差[\%] & 1145141919 \\ \hline
            \end{tabular}
        \end{flushleft}
    \end{table}

    %図の例
    \begin{figure}[H]
        \begin{center}
            \includegraphics[width = 8cm]{g1.png} %A4, 2段組の場合 width = 8cm が最適, homeDirectory/res/imageFile
            \caption{図のキャプションは下\label{fig:g1}}
        \end{center}
    \end{figure}

    % コードの例
    \begin{lstlisting}[basicstyle=\ttfamily\footnotesize, frame=single]
hogehoge
    \end{lstlisting}

    %数式の例
    %ちゃんとみせたいやつ
    \begin{align}
        \label{eq:e1}
        (PWC170) = \frac{170 - 96.85}{6.05} \times 10 = 120.90... \simeq 121 [W]\\ %\\で改行
        (PWC170) = \frac{170 - 96.85}{6.05} \times 10 = 120.90... \simeq 121 [W]
    \end{align}
    %文章中のやつ
    $ 変数x $はどうのこうの

    %本文の例
    私が取り上げる参考文献\cite{YJ1919}は臭い
    %空白の改行で新しい段落 

    その調査として用いたデータとして、表\ref{tb:t1}, 図\ref{fig:g1}, 式\ref{eq:e1}, 章節\ref{sec:s1}...\\
    ああああ %\\で改行

    %参考文献の出力
    \bibliography{main}
\end{document}